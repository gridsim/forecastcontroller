 
\documentclass{article}
\usepackage[affil-it]{authblk}
\usepackage{graphicx}
\usepackage[space]{grffile}
\usepackage{latexsym}
\usepackage{amsfonts,amsmath,amssymb}
\usepackage{url}
\usepackage[utf8]{inputenc}
\usepackage{hyperref}
\hypersetup{colorlinks=false,pdfborder={0 0 0}}
\usepackage{textcomp}
\usepackage{longtable}
\usepackage{multirow,booktabs}
\usepackage[a4paper, total={6in, 9in}]{geometry}


\begin{document}

\bibliographystyle{plain}




\begin{align*}
\begin{tabular}{llll}
min. & $\displaystyle{\sum_{s \in S}(c_s \cdot p_s + E \cdot e_s) + D \cdot diff}$  &  & (1) \\
{s.t.} & $\displaystyle{\sum_{s \in S} t^{in}_{final(s)} \ge t_{mean}*|S| - diff}$ &  & (2) \\
& $\displaystyle{ \theta_{s}^{in} = t^{in}_{init(s)} + \frac{p_s \cdot \sigma \cdot \delta}{C}}$ & $\displaystyle{ \forall s \in S}$ & (3) \\
& $\displaystyle{\theta_{s}^{out} = \frac{\delta \cdot \lambda \cdot (t^{out}_s - t^{in}_{final(s)})}{C}}$ &  $\displaystyle{ \forall s \in S}$ & (4) \\
& $\displaystyle{t^{in}_{final(s)} = \theta^{in}_s + \theta^{out}_s}$ & $\displaystyle{ \forall s \in S}$ & (5) \\
& $\displaystyle{t^{in}_{final(s)} - e_s \le t_{mean} + \frac{\epsilon}{2}}$ & $\displaystyle{ \forall s \in S}$ & (6) \\
& $\displaystyle{t^{in}_{final(s)} + e_s \ge t_{mean} - \frac{\epsilon}{2}}$ & $\displaystyle{ \forall s \in S}$ & (7) \\
& $\displaystyle{t^{in}_{init(1)}} = \Theta$ & & (8) \\
& $\displaystyle{t^{in}_{init(s)} = t^{in}_{final(s-1)}}$ & $\displaystyle{ \forall s \in S, s \ge 1}$ & (9) \\
\end{tabular}
\end{align*}

\noindent
\textbf{Données:} \\
\begin{tabular}{ l p{0.9\textwidth} }
$S$: & L'ensemble des créneaux pour une période de décision \\
$E$: & Le poids pour une température hors de la préférence utilisateur dans la fonction objectif \\
$D$: & Le poids pour un dépassement négatif de la température moyenne durant une période de décision \\
$c_s$: & Le coût du créneau $s$ \\
$t^{out}_{s}$: & La température extérieur du créneau $s$\\
$t_{mean}$: & La température de référence pour le bâtiment déterminé par l'utilisateur \\
$\epsilon$: & L'hystérisis de la température déterminé selon la préférence de l'utilisateur\\
$C$: & Capacité thermale du bâtiment \\
$\Theta$: & La température initiale du bâtiment avant de procéder à l'optimisation \\
$\sigma$: & L'efficacité de la pompe à chaleur \\
$\delta$: & Nombre de secondes dans un créneau \\
$\lambda$: & Conductivité thermale du bâtiment avec l'extérieur \\
\end{tabular} 
\vspace{0.3cm}

\noindent
\textbf{Variables de décision:} \\
\begin{tabular}{ l p{0.9\textwidth} }
$p_s$: & La décision on/off de l'appareil pour le créneau $s$. \\
$diff$: & Décision d'accepter ou non une température moyenne négative durant toute la période d'optimisation (incluant une forte pénalité). \\
$e_s$: & Le dépassement de la température durant le créneau $s$ par rapport à la préférence utilisateur. La température $t$ a été dépassée si  $t \ge t_{mean} + \frac {epsilon}{2}$ et $t \le t_{mean} - \frac {epsilon}{2}$. Si c'est le cas, une pénalité est calculée.  \\
\end{tabular} 
\vspace{0.3cm}

\noindent
\textbf{Variables de décisions intermédiaires:} \\
\begin{tabular}{ l p{0.9\textwidth} }
$t^{in}_{init(s)}$: & La température initiale au créneau $s$ avant l'influence météo et du radiateur.\\
$t^{in}_{final(s)}$: & La température finale au créneau $s$ après l'influence extérieur et du radiateur. \\
$\theta^{in}_{s}$: & Température influencée par les appareils et la capacité thermale (pompe à chaleur, radiateur,...) \\
$\theta^{out}_{s}$: & Température influencée par l'extérieur (température, couplage avec l'extérieur, ...).\\
  
\end{tabular} 
  
  

\end{document}

